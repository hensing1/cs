\documentclass[a4paper,12pt]{article}

\usepackage[ngerman]{babel}
\usepackage[T1]{fontenc}
\usepackage{amsmath}
\usepackage{amssymb}

\title{Lösungen für Übungsblatt 1}
\author{Henning Lehmann, Ayoub Errami}
\date{18.10.2022}

\begin{document}
\maketitle
\section{Aufgabe 1.1: Vereinfachen von Funktionen}

\begin{itemize}
	\item $g_1(n)=n^4$
	\item $g_2(n)=1$
	\item $g_3(n)=n^{3,5}$
	\item $g_4(n)=max(k_1,k_2)$
\end{itemize}

\section{Aufgabe 1.2: Algorithmus analysieren}
\subsection{(a)}
\subsubsection{Theorem}

Der Algorithmus gibt eine absteigend sortierte Permutation von A zurück.

\subsubsection{Beweis}

\begin{description}
	\item{Sei $\text{sort}(X)$ eine absteigend sortierte Permutation einer Zahlenfolge $X$.}
	\item{Sei $S_n(x)$ eine Zahlenfolge mit den $n$ größten Elementen aus einer Zahlenfolge $X$.}
	\item{Sei $U_n(A)$ der initiale Inhalt von $A$ ohne die $n$ größten Elemente.}
\end{description}

\paragraph{Invariante in Zeile 1:}
$$
A[1..i-1]=\text{sort}(S_{i-1}(A))
$$
$$
A[i..n]=U_{i-1}(A)
$$

\paragraph{Induktionsanfang i=1:}
$$
A[1..0]=\text{leere Zahlenfolge}=\text{sort}(s_0(A))
$$
$$
A[1..n]=U_0(A)
$$

\paragraph{Induktionsschritt}\mbox{}\\
\indent Angenommen die Invariante gilt für ein $i \ge 1$. \\
Im Schleifendurchlauf wird das größte Element aus $A[i..n]$ an die Stelle $i$ gesetzt, wobei alle kleineren Elemente in $A[i+1..n]$ verbleiben. \\
D.h. am Ende der Schleife gilt:
\begin{description}
	\item $A[1..i]=\text{sort}(s_i(A))$
	\item $A[i+1..n]=U_i(A)$
	\item $\Rightarrow$ die Invariante gilt auch für $i+1$.
\end{description}
Für $i=n-1$:
\begin{description}
	\item $A[1..n-1]=\text{sort}(s_{n-1}(A))$
	\item $A[n..n]=U_{n-1}(A)$
\end{description}
Eine Zahlenreihe aus $n$ Elementen ohne die $n-1$ größten Elemente enthält trivialerweise lediglich das kleinste Element, welches sich hierbei in $A[n]$ befindet. Da die übrigen Elemente sich sortiert in $A[1..n-1]$ befinden, folgt: \\
\indent Die Ausgabe des Algorithmus ist eine absteigend sortierte Permutation von A. $\hfill\square$

\subsection{(b)}
\begin{description}
	\item Objektvergleiche in Z.2: $n-i$.
	\item Objektvergleiche in Z.3: $1$.
	\item $\rightarrow$ Objektvergleiche pro Schleifendurchlauf: $n-i+1$.
\end{description}
Insgesamt:
\begin{equation}
\begin{split}
\sum_{i=1}^{n-1}(n-i+1) & = (n-1)*n-\frac{(n-1)*n}{2}+(n-1) \\
& = n^2-n-0,5n^2-0,5n+n-1 \\
& = 0,5n^2-0,5n-1 \in \Theta(n^2)
\end{split}
\end{equation}

\subsection{(c)}
\paragraph{Minimale Vertauschungen:}\mbox{}\\
\indent Bereits absteigend sortierte Zahlenfolge (z.B. $[5,4,3,2,1]$). \\
$\rightarrow$ 0 Vertauschungen, da sich das Maximum aus $A[i..n]$ immer an der Stelle $i$ befindet und daher in Z.3 nie $j \neq i$.

\paragraph{Maximale Vertauschungen:}\mbox{}\\
\indent Zahlenfolge, bei welcher das kleinste Element an erster Stelle steht, der Rest jedoch absteigend sortiert ist (z.B. $[1,5,4,3,2]$). \\
$\rightarrow n-1$ Vertauschungen, da das kleinste Element bei jedem Schleifendurchlauf einen Platz nach rechts getauscht wird, bis es nach $n-1$ Vertauschungen an seinem korrekt einsortierten Platz ankommt.

\section{1.3: O-Notation}

\begin{center}
\begin{tabular}{| r | c | c | c | c | c | c | c |}
\hline
& $s(n)$ & $log_2(n)$ & $2n$ & $3^n$ & $\frac{log_2(n)}{sqrt(n)}$ & $0,05$ & $ne^n$ \\ \hline
$s(n)$ & $\Theta$ & - & O & o & $\omega$ & $\Omega$ & o \\ \hline
$log_2(n)$ & & $\Theta$ & o & o & $\omega$ & $\omega$ & o \\ \hline
$2n$ & & & $\Theta$ & o & $\omega$ & $\omega$ & o \\ \hline
$3^n$ & & & & $\Theta$ & $\omega$ & $\omega$ & o \\ \hline
$\frac{log_2(n)}{sqrt(n)}$ & & & & & $\Theta$ & o & o \\ \hline
$0,05$ & & & & & & $\Theta$ & o \\ \hline
$ne^n$ & & & & & & & $\Theta$ \\ \hline
\end{tabular}
\end{center}
Um die restlichen Felder auszufüllen, orientiere man sich am gegenüberliegenden Feld der Diagonale: 
\begin{description}
	\item $f=o(g) \iff g=\omega(f)$
	\item $f=O(g) \iff g=\Omega(f)$
	\item $f=\Theta(g) \iff g=\Theta(f)$
	\item Wenn keine Beziehung zwischen $f$ und $g$, dann auch keine Beziehung zwischen $g$ und $f$.
\end{description}

\end{document}