\documentclass[a4paper,12pt]{article}

\usepackage[ngerman]{babel}
\usepackage[T1]{fontenc}
\usepackage{amsmath}
\usepackage{amssymb}
\usepackage[utf8]{inputenc}

\title{Lösungen für Übungsblatt 2}
\author{Henning Lehmann}
\date{\today}

\begin{document}
\maketitle
\section*{Aufgabe 2.2: Rekursionsgleichungen lösen}

\paragraph{Theorem:}
Die geschlossene Form von $S(n)$ lautet $S(n)=2 \cdot n!$ für $n \geq 2$.

\paragraph{Beweis:}
\subparagraph{Induktionsanfang:} 
\begin{equation}
\begin{split}
S(2) &= \sum\limits_{i=1}^1 i \cdot S(i)=1 \cdot S(1)=4 \\
& = 2 \cdot 2!=2 \cdot 2 \cdot 1=4.
\end{split}
\end{equation}

\subparagraph{Induktionsannahme:}$S(n)=\sum\limits_{i=1}^{n-1}i \cdot S(i)=2 \cdot n!$ für alle $n\geq 2$.

\subparagraph{Induktionsschritt:}
\begin{equation}
\begin{split}
S(n+1)=\sum_{i=1}^ni \cdot S(i) &= n \cdot S(n) + \sum_{i=1}^{n-1}i \cdot S(i) \\
&=n\cdot S(n)+2\cdot n! \\
&=n\cdot 2\cdot n!+2\cdot n! \\
&=(n+1)\cdot 2\cdot n! \\
&=2\cdot (n+1)!
\end{split}
\end{equation}
$\hfill\square$

\clearpage

\section*{Aufgabe 2.4: Multiplikation zweier Zahlen}
\subsection*{(a)}
\textit{Annahme:} Das Verschieben einer Zahl um $i$ Stellen liegt in $\Theta(1)$.

\paragraph{Lösung:} Die "`Schulmethode"' zum Multiplizieren zweier Zahlen liegt in $\Theta(n^2)$.

\paragraph{Begründung:} Das Multiplizieren von $x$ mit einer Stelle von $y$ benötigt eine Laufzeit von $c*n$, da $x$ $n$ Stellen hat. Das anschließende Verschieben liegt laut Annahme in $\Theta(1)$.

Da $y$ ebenfalls $n$ Stellen hat, passieren insgesamt $n$ Multiplikationen mit jeweils einer Laufzeit in $\Theta(n)$. Daher liegt die Zeit für die Ausführung aller Multiplikationen in $\Theta(n^2)$.\\

Die hierbei entstehenden Zwischenergebnisse haben eine Durchschnittliche Stellenzahl von $3/2 n$ (sie werden durchschnittlich um $1/2 n$ Stellen nach rechts verschoben). Die Addition von zwei dieser Zwischenergebnisse liegt ebenfalls in $\Theta(n)$ und findet erneut $n$-mal statt - insofern liegt die Laufzeit für die Addition ebenfalls in $\Theta(n^2)$.\\

Insgesamt liegt die Laufzeit der Schulmethode also in $\Theta(n^2)+\Theta(n^2)=\Theta(n^2)$.

\subsection*{(b)}
Anzugeben ist die Rekursionsgleichung für \textsc{CleverMult} in der allgemeinen Form $T(n)=aT(n/b)+f(n)$.\\

\noindent Für den beschriebenen Algorithmus ergeben sich folgende Parameter:
\begin{itemize}
	\item $a=3$, da es drei rekursive Aufrufe gibt.
	\item $b=2$, da sich die Problemgröße $n$ bei jedem Aufruf halbiert.
	\item $f(n)=c\cdot n=\Theta(n)$, da alle übrigen Operationen Additionen sind, welche gemäß Aufgabenstellung in $\Theta(n)$ liegen.
\end{itemize}

\noindent Die Rekursionsgleichung lautet also $T(n)=3T(n/2)+\Theta(n)$.\\

\noindent Da $f(n)\in O(n^{\log_23-\epsilon})$ mit $\epsilon = 1 > 0$, gilt laut Master-Theorem:
$$T(n)=\Theta(n^{\log_23})\approx \Theta(n^{1,58})$$

\end{document}